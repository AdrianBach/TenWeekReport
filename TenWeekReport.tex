\documentclass[12pt,a4paper]{article}
\usepackage[utf8]{inputenc}
\usepackage[english]{babel}
\usepackage{amsmath}
\usepackage{amsfonts}
\usepackage{amssymb}
\usepackage[left=2cm,right=2cm,top=2cm,bottom=2cm]{geometry}
\author{Adrian Bach}
\title{Using artificial intelligence to improve decision-making in conservation conflicts \\\medskip Ten weeks report}

% bibliography 
%\usepackage{natbib}
%\bibliographystyle{humannat}

\begin{document}
\maketitle

%\newpage
\tableofcontents

\newpage
\section{Conservation conflicts}

6th mass extinction? Ref: \\
One of the many possible reasons, intense competition/antagonism between human activity and wildlife. Ref: \\
Conservation as a way to tame this problem. Ref: \\

\subsection{Conservation}

Different kinds of conservation actions.
Protecting the whole ecosystem from human impact: protection areas, ?. Ref: Wilgen2011, Brainbridge2017.
Reaction (without preventing contact): culling control by monetary incentives (Ref: ).
Offsetting: implementing elsewhere what has been damaged somewhere. Ref: Dr Ascelin Gordon.\\
But conserving a species for its own sakes sometimes lead it to reach numbers being problematic for human activities. Ref: Redpath's book.

\subsection{Conflicts}

Definition of a conflict. Ref: Redpath's book.\\
Examples from ConFooBio. Ref: Redpath's book, redpath2013, mason2017.
Elsewhere: behr2017, glynatsi2018.\\
Resolving these conflicts require effective management strategies.
Management strategy: a sequence of actions on the system in order to achieve goals.\\
Yet, ecosystems dynamics are highly complex and interconnected, their evolution very difficult to anticipate.
Very few successful implementations. (Ref:)

\subsection{Problems faced in conflicts resolution}

\subsubsection{Complexity}

\subsubsection{Lack of data}

\subsubsection{Divergent interests between stakeholders}

\subsubsection{Rigidity}

Unexpectedly, there is little  evidence that bigger budgets make conservation easier or more effective (Ref: game2013conservation)

\subsection{More generally, uncertainty}
At four different levels

\section{Adaptive management}

\subsection{Purpose}

how it deals with certain problems. 

\subsection{Modelling}

need for it.\\
examples. Ref: schluter2012, rumpff2011 (dealt with uncertainty by implementing different scenarios), brainbridge2017.

\subsection{Limits}

critics from game2013.\\
Diversity, lack of common framework. Ref: schluter2012.\\
unresolved problems.

\subsection{MSE}

Schema.
All the problems it deals with. Ref: bunnefeld2011.\\
First implemented in fisheries than applied on terrestrial animals conservation. Ref: bunnefeld2011, bunnefeld2013.\\

For a proper implementation, need for human decision-making modelling, because main reason for failure. Ref: schluter2012.
 
\section{Decision-making modelling}

\subsection{Game theory}
brief description. Ref: myerson1997.\\
Definition of the key concepts like utility. Ref: myerson1997.

\subsection{Application to conservation conflicts}
Ref: colyvan2011, Glynasti2018 (emphasize on the fact that it's recent).\\

need for a model coupling all this.
\section{GMSE} Ref: duthie2018gmse

\subsection{Formalisation of MSE framework}

describe how it falls in MSE framework, how it deals with uncertainty at each level, consensus biases, long term foreseeing.\\
Explain clearly what it is meant for.\\

\subsection{Decision-making artificial intelligence}

Genetic algorithm. Very accessible worded explanation.\\
How is it suited to human decision-making?

\subsection{limits}

\subsubsection{Theoretical}

Agents act independently, which is very unlikely. REF???!!\\
Different types of conservation interests.
Would be interesting to implement.\\
Does not consider the do nothing option which is sometimes interesting.\\
Measure of conflict intensity?

\subsubsection{Computational}
Computing time when the number of stakeholders increases.\\
Lacks machine learning to be a proper artificial intelligence.\\
Parallelism in general.

\section{Research Questions}

They have to be very closely related to conservation (to avoid making it a mere modelling project)

\subsection{Case study: Geese}

Description. Ref: mason2017, brainbridge2017.
Its attributes (liked with the limits of GMSE). 

\subsection{Action threshold}

\subsection{Stakeholders behaviour}

\subsection{Interaction}

\section{Expected outputs}

\end{document}