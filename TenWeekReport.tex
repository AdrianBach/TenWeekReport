\documentclass[12pt,a4paper]{article}
\usepackage[utf8]{inputenc}
\usepackage[english]{babel}
\usepackage{amsmath}
\usepackage{amsfonts}
\usepackage{amssymb}
\usepackage[left=2cm,right=2cm,top=2cm,bottom=2cm]{geometry}
\author{Adrian Bach}
\title{Using artificial intelligence to improve decision-making in conservation conflicts \\\medskip Ten-week report}

% bibliography 
\usepackage{natbib}
\bibliographystyle{humannat}

\begin{document}
\maketitle

%\newpage
\tableofcontents

\newpage
\section{Conservation conflicts}

%6th mass extinction? Ref: \\
At the beginning of what could be a new mass extinction episode (LACKS REF), preserving the remaining Nature is a central concern for humanity.
Many convictions can justify this statement, but the fact is that we undeniably depend on the services ecosystems provide us. (no need for references, right?)
Pollination, soil enrichment, water treatment, among many more, are essential for food production for example.
However, the key for the permanence of ecosystems is diversity, as it assures a quick and dynamic response to change.
Note that this variety is often an inspiration for innovation in any kind of technology.
Thus, conservation of biodiversity have become a leading field in ecology.
%
%One of the many aspects, intense competition/antagonism between human activity and wildlife. Ref: \\
%Conservation as a way to tame this problem. Ref: \\

\subsection{Conservation}

Definition of conservation.
%Different kinds of conservation actions.
Conservation can be applied in many different policies.
It can be preventive, by establishing protected areas to preserve intact ecosystems from human impact \citep{vanwilgen2011critical, brainbridge2017goose},
%: protection areas, ?. Ref: Wilgen2011, Brainbridge2017.
or in direct reaction to a problem without preventing contact, \textit{e.g.} culling control by monetary incentives \citep{mason2017changing, cusack2017time}.
Another example is offsetting, meaning implementing elsewhere the natural features damaged somewhere by human activity \citep{gordon2011assessing}.
%But conserving a species for its own sakes sometimes lead it to reach numbers being problematic for human activities. Ref: Redpath's book.
Yet, completely successful implementation of conservation is scarce because of the numerous challenges it faces \citep{walters2011uncertainty, wilgen2011critical}.

\subsection{Problems faced in conservation}

\subsubsection{Complexity}
Conservation problems are recognized as highly complex and densely interconnected systems, including ecology, sociology, agronomy and climatology simultaneously.
Therefore, it is often impossible to isolate a causes of changes in the system.
%It makes their complete understanding out of our cognition.
%Ref: grimm1999individual, wilgen2011critical, runge2011uncertainty, schluter2013new,pretty much every article I've read.

\subsubsection{Lack of data}
These features make monitoring very expensive, time demanding, and often irrelevant, as the number of possible variables is too large.
Thus, conservation policies often lack data to account for their effectiveness, or to understand failure.
Furthermore, management is based on estimations of populations, which accuracy is challenging to assess according to the technique.
%Ref: rumff2011, all of the articles from the 2011 special edition.

\subsubsection{Rigidity}
Due to its complexity, predicting accurately the system's response to a change in the conservation policy is difficult, if not impossible.
It can results in reluctance to change, and in the maintain of inadequate conservation policies. REF
%Ref: geese case, apply a simple protection lead to population reaching conflictual numbers.

\subsection{More generally, uncertainty}
More generally, conservation faces uncertainty at many levels.

Other limits in keith2011uncertainty about politics and self-serving.
Unexpectedly, there is little  evidence that bigger budgets make conservation easier or more effective (Ref: game2013conservation)

Conservation had to be carried on despite these discouraging barriers, acting on protection while dealing with them.

\section{Adaptive management}

\subsection{Purpose}

%how it deals with certain problems. Acting while acquiring informations, learning from mistakes, closer to the system response to act as quickly as possible.
Adaptive management suggests to dynamically update the management policy according to the system behaviour.
This way, conservation can be better fitted to the system, and acting regularly allow to acquire informations on its response to change.
This allows to learn heuristically from the system, and react effectively.
%Uncertainty is an irreducible aspect of conservation, so it seems more relevant to act embracing it rather than waiting and keep on with an inadequate policy.
LACKS REF

\subsection{Limits}

critics from game2013.\\
other unresolved problems from above.

Even if a policy effectively protects a species from going extinct, mismanagement can lead populations to reach problematical numbers for human livelihood.

\section{Conflicts}

This is when conflicts arise, because people impacted by a protected population's growth are very likely to defect conservation policies, which can be threatening for the species persistence.
Definition of a conflict. Ref: Redpath's book.\\
Examples from ConFooBio. Ref: Redpath's book, redpath2013, mason2017.
Elsewhere: behr2017, glynatsi2018.
%Resolving these conflicts requires effective management strategies.
%Management strategy: a sequence of actions on the system in order to achieve goals.\\
%Yet, ecosystems dynamics are highly complex and interconnected, their evolution very difficult to anticipate.

\subsection{Divergent interests between stakeholders}
The inherent problem in conflicts is the divergence of stakeholders interests.
Farmers are usually way more interested in the yield they live on, then the survival of the species that is part of its decreasing.
Reaching of a consensus on a management target can therefore be an unproductive process.
Moreover it can prevent the situation from changing in a more equitable way \citep{peterson2005conservation}.

\subsection{MSE}

Management Strategy Evaluation is a framework that describes the process of Adaptive Management taking into account the conflicts it can rise.
It decomposes the problem in four main parts: manager's policy updating, user's harvest strategy, the species population and the mode of estimation of the population.
This structure isolates uncertainty at four main levels: decision-making under uncertainty, the reaction of the users, the population's response,
%links between components of the ecosystems,
and its estimation.
%politics
Schema.
%All the problems it deals with.
The circular structure is adapted to the heuristic updating of management policy.
Putting manager and user into different parts allow for goal-oriented behaviour, avoiding the need for consensus.
First implemented in fisheries than applied on terrestrial animals conservation. Ref: bunnefeld2011, bunnefeld2013.
Still need to establish targets.

\section{Modelling}

need for it, because conservation complexity is usually beyond our direct understanding. Decision helping tools, capable of considering more features than we do simultaneously.

\subsection{Different models used in literature}

examples. Ref: schluter2012, rumpff2011 (dealt with uncertainty by implementing different scenarios), brainbridge2017.

For a proper implementation, need for human decision-making modelling, because one of the main reasons for failures. Ref: schluter2012.

High diversity, lack of common framework. Ref: schluter2012.\\

\section{Decision-making modelling}

Most successful approach is game theory, initiated by ??? in ??? in their book ??? economics.

\subsection{Game theory}
brief description. Ref: myerson1997.\\
Definition of the key concepts like utility. Ref: myerson1997.

\subsection{Application to conservation conflicts}
Ref: colyvan2011, Glynasti2018 (emphasize on the fact that it's recent).\\

Recently developed a model coupling all this.
\section{GMSE} Ref: duthie2018gmse. emphasize on its novelty.

\subsection{Formalisation of MSE framework}

describe how it falls in MSE framework, how it deals with uncertainty at each level, consensus biases, long term foreseeing.
At four main levels: population dynamics, links between components of the ecosystems, type of estimation of the population, making the right decision without knowing precisely the outcomes, reaction of the users, politics, response of the system, ...\\
Explain clearly what it is meant for.

\subsection{First IBM in conservation}

"Individualbased
models (IBMs; also known as agent-based models) are
used to model the behaviour of a system at an individual level
by specifying simple rules for agents and allowing them to
interact. These models allow for complex behaviour to emerge
from simple interactions, though this comes at some cost to
interpretation and analysis." Ref: hamblin2012parctical.

\subsection{Decision-making artificial intelligence}

Genetic algorithm. Very accessible worded explanation. ref: hamblin2012practical\\
How is it suited to human decision-making?\\
Also used in solving game theoretical problems. Ref: Maynard Smith 1982.\\\
An example of interaction between IBM and genetic algorithm was in hamblin2009, where the parameter governing the interaction rules in foraging ants where allowed to evolve through a GA. 

\subsection{limits}

\subsubsection{Theoretical}

Agents act independently, which is very unlikely. REF???!!\\
Different types of conservation interests.
Would be interesting to implement.\\
Does not consider the do nothing option which is sometimes interesting.\\
Measure of conflict intensity?

\subsubsection{Computational}
Computing time when the number of stakeholders increases.\\
Lacks machine learning to be a proper artificial intelligence.\\
Parallelism in general.

\section{Research Questions}

They have to be very closely related to conservation (to avoid making it a mere modelling project)

\subsection{Case study: Geese}

Description. Ref: mason2017, brainbridge2017.
Its attributes (liked with the limits of GMSE). 

from now on always speak about goose, state and farmers to settle the problem in the context of geese

\subsection{Is waiting an interesting option for managers?}

Recently introduced in iacona2017evolutionary: optimal delay before using funds.

\subsubsection{Calculus of impact}
Dr Ascelin Gordon: the difference in the variable of interest when manager acts or do nothing.\\
Implementing the 'doing nothing' option in GMSE. Quantitatively assess the impact on repeated simulations. Mean deviation from conservation target at each time step.\\
According to duthie2018gmse. The number of extinctions over several simulations increases exponentially with the frequence of manager non-intervention. So, I suspect it is going to show that doing nothing is not sustainable as a policy in itself. Surprise!
But is it necessary to act as each time step, "as soon as possible"?

\subsubsection{Action threshold}
Currently in GMSE, a parameter sets the number of the manager's interventions per time step. Rigid, insensible to the situation, action if pop $>$ ou $<$ target, regardless of the size of this difference. acting even if the population if only a few individuals from the target.\\
First, fixed deviation from manager's target as an action threshold. has to be relative the the population size though, if population size is 100, 50 individuals missing or extra is very concerning, yet it is less if population size is tens of thousands. For example I could test thresholds of 1, 5, 10, \dots \% of target.\\ 
Quantitative assessment of the impact on the "quality" of the policy over repeated simulations for increasing threshold values: mean deviation from manager's target? Impact? Conflict reduction? number of extinctions? (Is there a chance that the result will be the same as the manager intervention frequency?)\\
Dynamic threshold? A function of deviation from target, or conflict intensity, that would modify the action threshold.\\
Waiting could imply saving a certain amount of budget for next step. Something that could also concern the users, maybe highlighting a best time to act.

\subsection{Including human values in adaptive management}

All conservationists have different goals according to their values, it would be interesting to implement this. Using the framework from futureconservation.org, I could use quantitative measures of conservationists values and attribute utilities accordingly, and assess if it indeed produce the expected results. Users yields and population size according to the value on nature - people axis, and to conservation - capitalism one. 

\subsection{Standing alone: how to manage a conservation conflict with interacting users?}

Possibly interaction among users, the information they perceive on their neighbours could influence their strategy.

\section{Expected outputs}

At least a paper on the action threshold, a talk in Newcastle, a poster in winter symposium, a participation at the Trondheim workshop. Updated version of GMSE. Field work for SNH, possibly in the policy team to which Aileen is close.

\newpage
\bibliography{10WRbib}
\nocite{*}

\end{document}